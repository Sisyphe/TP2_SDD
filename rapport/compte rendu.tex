\documentclass[12pt]{report}
%\documentclass[twoside, openright, 12pt]{report}
 
\usepackage[top=2cm, bottom=2cm, left=2cm, right=2cm]{geometry} % definition des marges

\usepackage[applemac]{inputenc}
\usepackage[T1]{fontenc}
\usepackage[francais]{babel} 
\usepackage{url}
\usepackage{graphicx}
\usepackage{titlepic}
\usepackage{titlesec, blindtext, color}
\usepackage{amsmath}
\usepackage{amssymb}
\usepackage{mathrsfs}
\usepackage{listings}
\usepackage{multirow}
\usepackage{caption}
\usepackage{minitoc}
\usepackage{perpage}
\usepackage{setspace}
\setstretch{1.5} % interligne de 1.5

%paragraph options
\setlength{\parindent}{15pt}
\setlength{\parskip}{6pt}

% enleve la mention chapitre
\titleformat{\chapter}[hang]{\bf\huge}{\thechapter}{2pc}{}

% Ligne orpheline ou veuve
\widowpenalty=10000 % empeche au maximum la coupure avant la derniere ligne 
\clubpenalty=10000 % empeche au maximum la coupure apres la premiere ligne 
\raggedbottom

\lstset{language=C, basicstyle=\scriptsize, numbers=left, numberstyle=\scriptsize, numbersep=12pt} 

\def\maketitle{
\thispagestyle{empty}

\noindent 
  G�rald Lelong \\
  No�l Martignoni \\
  G31 \\
 
  \vskip 6cm
  
  \begin{center}\leavevmode
  \Large  \textbf{Structure de donn�e} \\
  \rule{9cm}{1pt} \\
  \Huge Compte rendu TP2 \\
  \end{center}
      
  \cleardoublepage
  }  

% Le document --------------------------------------------------------------------------------------------------------
\begin{document}

\maketitle

\tableofcontents

\chapter{Pr�sentation du TP}

\section{Description}

L'objectif de ce TP et de cr�er une structure de donn�e permettant de stocker un dictionnaire sous la forme d'un arbre.

Le programme prend en argument un lien vers un fichier qui contient un arbre sous forme parenth�s�.

Ensuite un menu permet d'ins�rer un mot dans le dictionnaire cr��, d'afficher le dictionnaire ou de quitter le programme.

\section{La structure de donn�es}

Les donn�es sont compos�es de liste chain�es organis�es en lien vertical lien horizontal. Chaque �l�ment contient :
\begin{itemize}
\item une lettre (char)
\item un pointeur sur son fr�re
\item un pointeur sur son fils
\end{itemize}

Les algorithmes de cr�ation et de gestion des arbres n�cessite l'utilisation de files et de piles. C'est pourquoi nous avons cr�� deux autres structures de donn�es et les algorithmes qui permettent de les utiliser : une structure de gestion de pile et une structure de gestion de file.

\section{Organisation des ..}

\begin{description}
\item[main.c] Programme principal
\item[queue.h .c] Contient les fonctions de manipulation de \textit{files}
\item[stack.h .c] Contient les fonctions de manipulation de \textit{piles}
\item[tree.h .c] Contient les fonctions de manipulation d'\textit{arbres}
\item[dictionnary.h .c] Contient les fonctions de manipulation de \textit{dictionnaires}
\item[tools.c] Contient des fonctions utilitaires
\end{description}

\chapter{Programme C}

\section{main.c}

\lstinputlisting[language=C]{../main.c}

\section{queue.h}

\lstinputlisting[language=C]{../queue.h}

\section{queue.c}

\lstinputlisting[language=C]{../queue.c}

\section{stack.h}

\lstinputlisting[language=C]{../stack.h}

\section{stack.c}

\lstinputlisting[language=C]{../stack.c}

\section{tree.h}

\lstinputlisting[language=C]{../tree.h}

\section{tree.c}

\lstinputlisting[language=C]{../tree.c}

\section{dictionnary.h}

\lstinputlisting[language=C]{../dictionnary.h}

\section{dictionnary.c}

\lstinputlisting[language=C]{../dictionnary.c}

\section{tools.h}

\lstinputlisting[language=C]{../tools.h}



\chapter{Compilation et tests}

\section{Makefile}

\lstinputlisting[language=make]{../Makefile}

\section{Jeux de test}

\subsection{Les files}

\subsubsection{Enfilage dans une file vide}

\subsubsection{D�filage d'une file vide}

\subsubsection{Enfilage dans une file pleine}

\subsubsection{Gestion des indices}



\subsection{Les piles}

\subsubsection{Empilement dans une pile vide}

\subsubsection{D�pilement dans une file vide}

\subsubsection{Empilement dans une file pleine}



\subsection{Les dictionnaires}

\subsubsection{Cr�ation d'un arbre vide (buildtree)}

\subsubsection{Cr�ation d'un arbre simple (buildtree)}

\subsubsection{Insertion simple (insertWord)}

\subsubsection{Insertion mot dans arbre vide (insertWord)}

\subsubsection{Insertion mot d�j� pr�sent dans l'arbre (insertWord)}

\subsubsection{Insertion mot  non pr�sent dans l'arbre mais dont les lettres sont d�j� pr�sentes (insertWord)}

Exemple insertion du mot \textit{gris} dans un arbre qui contient d�j� le mot \textit{grison}

\end{document}






















